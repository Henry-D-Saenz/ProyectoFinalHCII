\documentclass{article}

%%%%
% PLOTS mapas y conglomerados
%%%%


\usepackage[utf8]{inputenc}
\usepackage{longtable}
\usepackage{authblk}
\usepackage{adjustbox}
\usepackage{natbib}


\title{LOS INDICES EN COLOMBIA}
% autores
\renewcommand\Authand{, y }
\author[1]{\normalsize Henry D Saenz}
\author[2]{\normalsize Laura M Montoya}

\affil[1,2]{\small  Universidad de los Andes\\
\texttt{{hd.saenz10,lm.montoya10}@uniandes.edu.col}}
\affil[1,2]{\small Ecole des Mines de Nantes\\
\texttt{{hsaenz16,lmontoya16}@imt-atlantique.fr}}

\date{29 de Junio de 2018}

%%%%
\usepackage{Sweave}
\begin{document}
\Sconcordance{concordance:Proyecto_Final_Introduccion.tex:Proyecto_Final_Introduccion.Rnw:%
1 28 1 1 0 33 1}


\maketitle


\begin{abstract}
Este es mi primer trabajo en exploracion y modelamiento de indices usando LATEX. Este trabajo lo he hecho bajo la filosofÃ<U+00AD>a de trabajo replicable. 

Colombia es un pa<ed>s... bla bla bla

Quisimos explorar el Indice de Desarrollo Humano en el pa<ed>s por regiones. Se desea ver si las regiones tienen un IDH parecido o hay mucha igualdad en el territorio. 
\end{abstract}

\section*{Introducci<f3>n}

Aqui les presento mi investigacion sobre diversos indices sociales en el mundo. Los indices los conseguÃ<U+00AD> de wikipedia, espero que les gusten mucho. Aqui les presento mi investigacion sobre diversos indices sociales en el mundo. Los indices los conseguÃ<U+00AD> de wikipedia, espero que les gusten mucho.Aqui les presento mi investigacion sobre diversos indices sociales en el mundo. Los indices los conseguÃ<U+00AD> de wikipedia, espero que les gusten mucho.Aqui les presento mi investigacion sobre diversos indices sociales en el mundo. Los indices los conseguÃ<U+00AD> de wikipedia, espero que les gusten mucho.
Aqui les presento mi investigacion sobre diversos indices sociales en el mundo. Los indices los conseguÃ<U+00AD> de wikipedia, espero que les gusten mucho.Aqui les presento mi investigacion sobre diversos indices sociales en el mundo. Los indices los conseguÃ<U+00AD> de wikipedia, espero que les gusten mucho.Aqui les presento mi investigacion sobre diversos indices sociales en el mundo. Los indices los conseguÃ<U+00AD> de wikipedia, espero que les gusten mucho.

Comencemos viendo que hay en la sección \ref{univariada} en la página \pageref{univariada}.

\clearpage



\section{Exploración Univariada}\label{univariada}

En esta sección exploro cada índice. En esta sección exploro cada índice. En esta sección exploro cada índice. En esta sección exploro cada índice. En esta sección exploro cada índice. En esta sección exploro cada índice. En esta sección exploro cada índice. En esta sección exploro cada índice. En esta sección exploro cada índice.



\begin{Schunk}
\begin{Soutput}
'data.frame':	32 obs. of  6 variables:
 $ IDH                : num  0.879 0.867 0.865 0.849 0.842 0.839 0.837 0.835 0.834 0.832 ...
 $ Departamento       : chr  "Santander" "Casanare" "Valle del Cauca" "Antioquia" ...
 $ Población.Cabecera: int  1587972 281548 4169553 5262172 742812 761658 10070801 2438533 56487 506254 ...
 $ Población.Resto   : int  502867 93701 586560 1428858 539251 206109 914484 107391 21926 68756 ...
 $ Población.Total   : int  2090839 375249 4756113 6691030 1282063 967767 10985285 2545924 78413 575010 ...
 $ DepartamentoNorm   : chr  "Santander" "Casanare" "Valle del Cauca" "Antioquia" ...
\end{Soutput}
\end{Schunk}

No podemos hacer tabla de frecuencias, entonces sacamos solamente los estadisticos:

% Table created by stargazer v.5.2.2 by Marek Hlavac, Harvard University. E-mail: hlavac at fas.harvard.edu
% Date and time: vie., jun. 29, 2018 - 5:16:59 p.m.
\begin{table}[!htbp] \centering 
  \caption{Medidas estadísticas} 
  \label{stats} 
\begin{tabular}{@{\extracolsep{5pt}}lcc} 
\\[-1.8ex]\hline 
\hline \\[-1.8ex] 
Statistic & \multicolumn{1}{c}{N} & \multicolumn{1}{c}{Median} \\ 
\hline \\[-1.8ex] 
Población.Cabecera & 32 & 717,197 \\ 
Población.Resto & 32 & 268,111.5 \\ 
Población.Total & 32 & 1,028,429 \\ 
\hline \\[-1.8ex] 
\end{tabular} 
\end{table} 





\endinput

\section{Exploracion Bivariada}\label{bivariada}

En este trabajo estamos interesados en el impacto de la poblacion en el IDH, veamos IDH con cada uno:


% Table created by stargazer v.5.2.2 by Marek Hlavac, Harvard University. E-mail: hlavac at fas.harvard.edu
% Date and time: Fri, Jun 29, 2018 - 8:31:31 PM
\begin{table}[!htbp] \centering 
  \caption{Correlacion del IDH con las demas variables} 
  \label{corrDem} 
\begin{tabular}{@{\extracolsep{5pt}} cc} 
\\[-1.8ex]\hline 
\hline \\[-1.8ex] 
cabeLog & restoLog \\ 
\hline \\[-1.8ex] 
$0.487$ & $0.177$ \\ 
\hline \\[-1.8ex] 
\end{tabular} 
\end{table} 

Veamos la correlacion entre las variables independientes:


% Table created by stargazer v.5.2.2 by Marek Hlavac, Harvard University. E-mail: hlavac at fas.harvard.edu
% Date and time: Fri, Jun 29, 2018 - 8:31:34 PM
\begin{table}[!htbp] \centering 
  \caption{Correlacion entre variables independientes} 
  \label{corrTableX} 
\begin{tabular}{@{\extracolsep{5pt}} ccc} 
\\[-1.8ex]\hline 
\hline \\[-1.8ex] 
 & cabeLog & restoLog \\ 
\hline \\[-1.8ex] 
cabeLog & 1 &  \\ 
restoLog & 0.84 & 1 \\ 
\hline \\[-1.8ex] 
\end{tabular} 
\end{table} 
Lo visto en la Tabla \ref{corrTableX} se refuerza claramente en la Figura \ref{corrPlotX}.

\begin{figure}[h]
\centering
\begin{adjustbox}{width=7cm,height=7cm,clip,trim=1.5cm 0.5cm 0cm 1.5cm}
\includegraphics{Proyecto_Final_bivariada-corrPlotX}
\end{adjustbox}
\caption{correlacion entre predictores}
\label{corrPlotX}
\end{figure}
\endinput

%\section{Modelos de Regresion}

Finalmente, vemos los modelos propuestos. Primero sin poblacion resto, luego con esa: Los resultados se muestran en la Tabla \ref{regresiones} de la pagina \pageref{regresiones}.

  
  
  
% Table created by stargazer v.5.2.2 by Marek Hlavac, Harvard University. E-mail: hlavac at fas.harvard.edu
% Date and time: Fri, Jun 29, 2018 - 8:02:32 PM
\begin{table}[!htbp] \centering 
  \caption{Modelos de Regresion} 
  \label{regresiones} 
\begin{tabular}{@{\extracolsep{5pt}}lcc} 
\\[-1.8ex]\hline 
\hline \\[-1.8ex] 
 & \multicolumn{2}{c}{\textit{Dependent variable:}} \\ 
\cline{2-3} 
\\[-1.8ex] & \multicolumn{2}{c}{IDH} \\ 
\\[-1.8ex] & (1) & (2)\\ 
\hline \\[-1.8ex] 
 cabeLog & 0.013$^{***}$ & 0.031$^{***}$ \\ 
  & (0.004) & (0.007) \\ 
  & & \\ 
 restoLog &  & $-$0.030$^{***}$ \\ 
  &  & (0.010) \\ 
  & & \\ 
 Constant & 0.634$^{***}$ & 0.766$^{***}$ \\ 
  & (0.055) & (0.065) \\ 
  & & \\ 
\hline \\[-1.8ex] 
Observations & 32 & 32 \\ 
R$^{2}$ & 0.238 & 0.425 \\ 
Adjusted R$^{2}$ & 0.212 & 0.385 \\ 
Residual Std. Error & 0.037 (df = 30) & 0.033 (df = 29) \\ 
F Statistic & 9.347$^{***}$ (df = 1; 30) & 10.706$^{***}$ (df = 2; 29) \\ 
\hline 
\hline \\[-1.8ex] 
\textit{Note:}  & \multicolumn{2}{r}{$^{*}$p$<$0.1; $^{**}$p$<$0.05; $^{***}$p$<$0.01} \\ 
\end{tabular} 
\end{table}   
  Como se vio en la Tabla \ref{regresiones}, cuando esta presente el \emph{indice de libertad mundial}, el \emph{Indice de libertad de prensa} pierde significancia.

\clearpage

\section{Exploracion Espacial}

Como acabamos de ver en la Tabla \ref{regresiones} en la pagina \pageref{regresiones}, si quisieras sintetizar la multidimensionalidad de nuestros indicadores, podramos usar tres de las cuatro variables que tenemos (un par de las originales tiene demasiada correlacion). 

%Asi, propongo que calculemos conglomerados de paises usando toda la informacion de tres de los indicadores. Como nuestras variables son ordinales utilizaremos un proceso de conglomeracion donde las distancia seran calculadas usando la medida {\bf gower} propuestas en \cite{gower_general_1971}, y para los enlazamientos usaremos la tecnica de {\bf medoides} segun \cite{reynolds_clustering_2006}. Los tres conglomerados se muestran en la Figura \ref{clustmap}.






\begin{figure}[h]
\centering
\begin{adjustbox}{width=20cm,height=15cm,clip,trim=22cm 0cm 0cm 0cm}
[1] 3 1 2  Group.1       IDH  cabeLog restoLog
1       1 0.8406154 14.12024 12.64415
2       2 0.7710833 13.28360 12.89513
3       3 0.7825714 10.58974 10.60684\includegraphics{Proyecto_Final_regresion-plotMapf}
\end{adjustbox}
\caption{Paises conglomerados segun sus indicadores sociopolaticos}\label{clustmap}
\end{figure}



\endinput



\bibliographystyle{apalike}
\renewcommand{\refname}{Bibliografia}
\bibliography{Colombia}

\end{document}
